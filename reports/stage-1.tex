\documentclass[a4paper]{article}
\usepackage{listings}
\usepackage{ctex}
\usepackage{xcolor}
\usepackage{graphicx}
\usepackage{float}
\usepackage{parskip}

\setlength{\parskip}{6pt}

\author{梁业升 2019010547(计03)}

\begin{document}

% GitHub styles
\definecolor{keyword}{HTML}{CF222E}
\definecolor{comment}{HTML}{6E7781}
\definecolor{string}{HTML}{0A3069}

\lstset{
    commentstyle=\color{comment},
    keywordstyle=\color{keyword},
    stringstyle=\color{string},
    basicstyle=\ttfamily\small,
    breakatwhitespace=false,
    breaklines=true,
    captionpos=b,
    keepspaces=true,
    showspaces=false,
    showstringspaces=false,
    showtabs=false,
}

\title{MiniDecaf Stage 1 Report}

\maketitle

\section{实验内容}

\subsection{词法语法分析}

词法分析:在 \texttt{scanner.l} 中添加 \texttt{BNOT}、\texttt{LNOT} 等运算符的词法:

\begin{lstlisting}
// scanner.l
"~"          { return yy::parser::make_BNOT(loc); }
"!"          { return yy::parser::make_LNOT(loc); }
"*"          { return yy::parser::make_TIMES(loc); }
"/"          { return yy::parser::make_SLASH(loc); }
"%"          { return yy::parser::make_MOD(loc); }
"=="         { return yy::parser::make_EQU(loc); }
"!="         { return yy::parser::make_NEQ(loc); }
"<"          { return yy::parser::make_LT(loc); }
"<="         { return yy::parser::make_LEQ(loc); }
">"          { return yy::parser::make_GT(loc); }
">="         { return yy::parser::make_GEQ(loc); }
"&&"         { return yy::parser::make_AND(loc); }
"||"         { return yy::parser::make_OR(loc); }
\end{lstlisting}

语法分析:在 \texttt{parser.y} 中,分别参照一元运算符 \texttt{MINUS} 和二元运算符 \texttt{PLUS} 的声明添加所实现运算符的语法:

\begin{lstlisting}
// parser.y
Expr: ...
    /* Unary */
    | MINUS Expr    %prec NEG
        { $$ = new ast::NegExpr($2, POS(@1)); }
    | BNOT Expr     %prec BNOT
        { $$ = new ast::BitNotExpr($2, POS(@1)); }
    | LNOT Expr     %prec LNOT
        { $$ = new ast::NotExpr($2, POS(@1)); }
    /* Binary */
    | Expr PLUS Expr
        { $$ = new ast::AddExpr($1, $3, POS(@2)); }
    | Expr MINUS Expr
        { $$ = new ast::SubExpr($1, $3, POS(@2)); }
    | Expr TIMES Expr
        { $$ = new ast::MulExpr($1, $3, POS(@2)); }
    | Expr SLASH Expr
        { $$ = new ast::DivExpr($1, $3, POS(@2)); }
    | Expr MOD Expr
        { $$ = new ast::ModExpr($1, $3, POS(@2)); }
    | Expr EQU Expr
        { $$ = new ast::EquExpr($1, $3, POS(@2)); }
    | Expr NEQ Expr
        { $$ = new ast::NeqExpr($1, $3, POS(@2)); }
    | Expr LT Expr
        { $$ = new ast::LesExpr($1, $3, POS(@2)); }
    | Expr GT Expr
        { $$ = new ast::GrtExpr($1, $3, POS(@2)); }
    | Expr LEQ Expr
        { $$ = new ast::LeqExpr($1, $3, POS(@2)); }
    | Expr GEQ Expr
        { $$ = new ast::GeqExpr($1, $3, POS(@2)); }
    | Expr AND Expr
        { $$ = new ast::AndExpr($1, $3, POS(@2)); }
    | Expr OR Expr
        { $$ = new ast::OrExpr($1, $3, POS(@2)); }
\end{lstlisting}

顺便一提,框架给出的的三元运算符的语法 \texttt{Expr QUESTION Expr COLON Expr} 存在 shift/reduce 冲突:

\begin{lstlisting}
frontend/parser.y: warning: shift/reduce conflict on token "+" [-Wcounterexamples]
  Example: Expr "?" Expr ":" Expr • "+" Expr
  Shift derivation
    Expr
    ↳ 41: Expr "?" Expr ":" Expr
                            ↳ 28: Expr • "+" Expr
  Reduce derivation
    Expr
    ↳ 28: Expr                           "+" Expr
        ↳ 41: Expr "?" Expr ":" Expr •
\end{lstlisting}

使用 \texttt{\%prec} 显式指出此表达式的优先级即可解决:

\begin{lstlisting}
Expr QUESTION Expr COLON Expr %prec QUESTION
\end{lstlisting}

\section{中间代码生成}

与上面类似,一元/二元运算符参考一元运算符 \texttt{MINUS} 和二元运算符 \texttt{PLUS},在各个涉及的 Visitor(\texttt{Translation} 和 \texttt{SemPass2})中添加对应的 \texttt{visit} 函数即可:

\begin{lstlisting}[language=c++]
// Translation.cpp
/* Unary */
void Translation::visit(ast::BitNotExpr *e) {
    e->e->accept(this);
    e->ATTR(val) = tr->genBNot(e->e->ATTR(val));
}
...
/* Binary */
void Translation::visit(ast::SubExpr *e) {
    e->e1->accept(this);
    e->e2->accept(this);
    e->ATTR(val) = tr->genSub(e->e1->ATTR(val), e->e2->ATTR(val));
}
...
\end{lstlisting}

\begin{lstlisting}[language=c++]
// type_check.cpp
/* Unary */
void SemPass2::visit(ast::BitNotExpr *e) {
    e->e->accept(this);
    expect(e->e, BaseType::Int);
    e->ATTR(type) = BaseType::Int;
}
...
/* Binary */
void SemPass2::visit(ast::SubExpr *e) {
    e->e1->accept(this);
    expect(e->e1, BaseType::Int);
    e->e2->accept(this);
    expect(e->e2, BaseType::Int);
    e->ATTR(type) = BaseType::Int;
}
...
\end{lstlisting}

\section{目标代码生成}

部分中间代码有直接对应的的(伪)汇编代码,直接转换即可:

\begin{itemize}
    \item \texttt{BNOT}:\texttt{not}
    \item \texttt{LNOT}:\texttt{seqz}
    \item \texttt{SUB}:\texttt{sub}
    \item \texttt{MUL}:\texttt{mul}
    \item \texttt{DIV}:\texttt{div}
    \item \texttt{MOD}:\texttt{rem}
    \item \texttt{LES}:\texttt{slt}
    \item \texttt{GTR}:\texttt{sgt}
\end{itemize}

其余中间代码没有对应的汇编,需要组合多条汇编代码实现:

\begin{itemize}
    \item \texttt{EQU}:\texttt{sub} + \texttt{seqz}
    \item \texttt{NEQ}:\texttt{sub} + \texttt{snez}
    \item \texttt{LEQ}:\texttt{sgt} + \texttt{seqz}
    \item \texttt{GEQ}:\texttt{slt} + \texttt{seqz}
    \item \texttt{LAND}:\texttt{and} + \texttt{snez}
    \item \texttt{LOR}:\texttt{or} + \texttt{snez}
\end{itemize}

\section{思考题}

\begin{enumerate}
    \item \textbf{Step 2}:\texttt{-\textasciitilde2147483647}
    \item \textbf{Step 3}:\begin{enumerate}
        \item \textbf{x86-64}:\texttt{floating point exception (core dumped)}
        \item \textbf{riscv32}:\texttt{-2047483648}
    \end{enumerate}
    \item \textbf{Step 4}:短路时第二个表达式无需计算,可以给程序编写提供便利,如第二个表达式依赖于第一个表达式所满足的某些条件。
\end{enumerate}

\end{document}