\documentclass[a4paper]{article}
\usepackage{listings}
\usepackage{ctex}
\usepackage[svgnames]{xcolor}
\usepackage{graphicx}
\usepackage{float}
\usepackage{parskip}

\setlength{\parskip}{6pt}

\author{梁业升 2019010547(计03)}

\lstdefinelanguage{diff} {
    basicstyle=\ttfamily\small,
    morecomment=[f][\color{diffstart}]{@@},
    morecomment=[f][\color{diffincl}]{+\ },
    morecomment=[f][\color{diffrem}]{-\ },
  }

\begin{document}

% GitHub styles
\definecolor{keyword}{HTML}{CF222E}
\definecolor{comment}{HTML}{6E7781}
\definecolor{string}{HTML}{0A3069}
\definecolor{diffstart}{named}{Grey}
\definecolor{diffincl}{named}{Green}
\definecolor{diffrem}{named}{OrangeRed}

\lstset{
    commentstyle=\color{comment},
    keywordstyle=\color{keyword},
    stringstyle=\color{string},
    basicstyle=\ttfamily\small,
    breakatwhitespace=false,
    breaklines=true,
    captionpos=b,
    keepspaces=true,
    showspaces=false,
    showstringspaces=false,
    showtabs=false,
}

\title{MiniDecaf Stage 3 Report}

\maketitle

\section{实验内容}

\underline{对于 Step 7 的作用域与块,实验框架已经实现,在此不再赘述。}

\subsection{词法语法分析}

While 语句增加 Do-while 支持:

\begin{lstlisting}[language=diff]
  WhileStmt : WHILE LPAREN Expr RPAREN Stmt
-     { $$ = new ast::WhileStmt($3, $5, POS(@1)); }
+     { $$ = new ast::WhileStmt($3, $5, /*hasDo=*/false, POS(@1)); }
+ | DO Stmt WHILE LPAREN Expr RPAREN SEMICOLON
+     { $$ = new ast::WhileStmt($5, $2, /*hasDo=*/true, POS(@1)); }
+ ;
\end{lstlisting}

相应地,\texttt{WhileStmt} 中增加一个成员 \texttt{bool hasDo},表示是否是 Do-while 循环。

增加 For 循环:

\begin{lstlisting}[language=diff]
+ ForStmt : FOR LPAREN Expr SEMICOLON Expr SEMICOLON Expr RPAREN Stmt
+     { $$ = new ast::ForStmt($3, $5, $7, $9, POS(@1)); }
+ | FOR LPAREN SEMICOLON Expr SEMICOLON Expr RPAREN Stmt
+     { $$ = new ast::ForStmt((ast::Expr*)nullptr, $4, $6, $8, POS(@1)); }
+ | FOR LPAREN Expr SEMICOLON SEMICOLON Expr RPAREN Stmt
+     { $$ = new ast::ForStmt($3, nullptr, $6, $8, POS(@1)); }
+ | FOR LPAREN Expr SEMICOLON Expr SEMICOLON RPAREN Stmt
+     { $$ = new ast::ForStmt($3, $5, nullptr, $8, POS(@1)); }
+ | FOR LPAREN SEMICOLON Expr SEMICOLON RPAREN Stmt
+     { $$ = new ast::ForStmt((ast::Expr*)nullptr, $4, nullptr, $7, POS(@1)); }
+ | FOR LPAREN Expr SEMICOLON SEMICOLON RPAREN Stmt
+     { $$ = new ast::ForStmt($3, nullptr, nullptr, $7, POS(@1)); }
+ | FOR LPAREN SEMICOLON SEMICOLON Expr RPAREN Stmt
+     { $$ = new ast::ForStmt((ast::Expr*)nullptr, nullptr, $5, $7, POS(@1)); }
+ | FOR LPAREN SEMICOLON SEMICOLON RPAREN Stmt
+     { $$ = new ast::ForStmt((ast::Expr*)nullptr, nullptr, nullptr, $6, POS(@1)); }
+ | FOR LPAREN DeclStmt Expr SEMICOLON Expr RPAREN Stmt
+     { $$ = new ast::ForStmt($3, $4, $6, $8, POS(@1)); }
+ | FOR LPAREN DeclStmt Expr SEMICOLON RPAREN Stmt
+     { $$ = new ast::ForStmt($3, $4, nullptr, $7, POS(@1)); }
+ | FOR LPAREN DeclStmt SEMICOLON Expr RPAREN Stmt
+     { $$ = new ast::ForStmt($3, nullptr, $5, $7, POS(@1)); }
+ | FOR LPAREN DeclStmt SEMICOLON RPAREN Stmt
+     { $$ = new ast::ForStmt($3, nullptr, nullptr, $6, POS(@1)); }
\end{lstlisting}

增加 Continue:

\begin{lstlisting}[language=diff]
  Stmt : ReturnStmt {$$ = $1;} |
      ...
+     CONTINUE SEMICOLON
+       { $$ = new ast::ContStmt(POS(@1)); } |
      ...
\end{lstlisting}

\subsection{符号表构建}

在第一个 Pass,对于 \texttt{ForStmt} 节点,如在括号中声明变量,则为其创建一个本地作用域。需要注意的是此作用域在循环体内可见。

\begin{lstlisting}[language=c++]
void SemPass1::visit(ast::ForStmt *s) {
    // Scope for the loop variable
    if (s->initDecl != nullptr) {
        Scope *declScope = new LocalScope();
        s->ATTR(decl_scope) = declScope;
        scopes->open(declScope);
        s->initDecl->accept(this);
    }
    if (s->cond)
        s->cond->accept(this);
    if (s->update)
        s->update->accept(this);
    s->body->accept(this);
    if (s->initDecl != nullptr)
        scopes->close();
}
\end{lstlisting}

\texttt{WhileStmt} 的符号表构建过程沿用框架给出的实现即可。

\subsection{类型检查}

针对新增的表达式 \texttt{ForStmt} 增加类型检查(其余新增的表达式在框架中已给出),需要注意括号内表达式为空的情形:

\begin{lstlisting}[language=c++]
void SemPass2::visit(ast::ForStmt *s) {
    if (s->initExpr) {
        s->initExpr->accept(this);
        if (!s->initExpr->ATTR(type)->equal(BaseType::Int))
            issue(s->initExpr->getLocation(), new BadTestExprError());
    }
    if (s->initDecl) {
        scopes->open(s->ATTR(decl_scope));
        s->initDecl->accept(this);
    }
    if (s->cond) {
        s->cond->accept(this);
        if (!s->cond->ATTR(type)->equal(BaseType::Int))
            issue(s->cond->getLocation(), new BadTestExprError());
    }
    if (s->update) {
        s->update->accept(this);
        if (!s->update->ATTR(type)->equal(BaseType::Int))
            issue(s->update->getLocation(), new BadTestExprError());
    }
    s->body->accept(this);
    if (s->initDecl)
        scopes->close();
}
\end{lstlisting}

\subsection{翻译为中间代码}

For 循环对应的三地址码如下:

\begin{lstlisting}
    ... # 初始化语句/表达式
L1:
    ... # 条件判断
    JZERO xx, L3
    ... # 循环体语句
L2:
    ... # 更新语句
    JUMP L1
L3:
\end{lstlisting}

对应的翻译代码如下,注意需要将当前循环体内的 \texttt{L3}、\texttt{L2} 分别“入栈” \texttt{break\_label}、 \texttt{continue\_label}:

\begin{lstlisting}[language=c++]
void Translation::visit(ast::ForStmt *s) {
    Label l1 = tr->getNewLabel();
    Label l2 = tr->getNewLabel();
    Label l3 = tr->getNewLabel();

    // Visit init statement / expression
    if (s->initDecl)
        s->initDecl->accept(this);
    else if (s->initExpr)
        s->initExpr->accept(this);

    Label old_break = current_break_label;
    Label old_continue = current_continue_label;
    current_break_label = l3;
    current_continue_label = l2;

    tr->genMarkLabel(l1);

    if (s->cond) {
        s->cond->accept(this);
        tr->genJumpOnZero(l3, s->cond->ATTR(val));
    }
    if (s->body)
        s->body->accept(this);
    tr->genMarkLabel(l2);
    if (s->update)
        s->update->accept(this);
    tr->genJump(l1);
    tr->genMarkLabel(l3);

    current_break_label = old_break;
    current_continue_label = old_continue;
}
\end{lstlisting}

对于 \texttt{WhileStmt},由于增加了 Do-while,需要根据 \texttt{s->hasDo} 进行相应的调整(即将条件判断的 TAC 置于不同位置):

\begin{lstlisting}[language=c++]
void Translation::visit(ast::WhileStmt *s) {
    Label L1 = tr->getNewLabel();
    Label L2 = tr->getNewLabel();

    Label old_break = current_break_label;
    Label old_continue = current_continue_label;
    current_break_label = L2;
    current_continue_label = L1;

    tr->genMarkLabel(L1);
    s->condition->accept(this);
    tr->genJumpOnZero(L2, s->condition->ATTR(val));
    if (!s->hasDo) {
        s->condition->accept(this);
        tr->genJumpOnZero(L2, s->condition->ATTR(val));
    }
    s->loop_body->accept(this);
    if (s->hasDo) {
        s->condition->accept(this);
        tr->genJumpOnZero(L2, s->condition->ATTR(val));
    }
    tr->genJump(L1);
    tr->genMarkLabel(L2);

    current_break_label = old_break;
    current_continue_label = old_continue;
}
\end{lstlisting}

\section{思考题}

\begin{enumerate}
    \item \textbf{Step 7}:

          由 Mind 生成的汇编代码经精简后如下:
          \begin{lstlisting}
----------------------- A
main:
    sw    ra, -4(sp)
    sw    fp, -8(sp)
    mv    fp, sp
    addi  sp, sp, -8
    li    t0, 2
    add   t1, zero, t0
    li    t2, 3
    slt   s1, t1, t2
    beqz  s1, __LL1
----------------------- B
    li    t0, 3
    add   t1, zero, t0
    mv    a0, t1
    mv    sp, fp
    lw    ra, -4(fp)
    lw    fp, -8(fp)
    ret
----------------------- C
__LL1:
    li    t0, 0
    mv    a0, t0
    mv    sp, fp
    lw    ra, -4(fp)
    lw    fp, -8(fp)
    ret
-----------------------
\end{lstlisting}

          控制流图如下:
          \begin{lstlisting}
A --> B
 \
  -> C
\end{lstlisting}
    \item \textbf{Step 8}:第一种更好。假设执行一次循环体,循环体、条件判断均只有 1 条指令,则第一种执行 6 条指令,第二种执行 7 条指令。
\end{enumerate}
\end{document}