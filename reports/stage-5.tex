\documentclass[a4paper]{article}
\usepackage{listings}
\usepackage{ctex}
\usepackage[svgnames]{xcolor}
\usepackage{graphicx}
\usepackage{float}
\usepackage{parskip}

\setlength{\parskip}{6pt}

\author{梁业升 2019010547(计03)}

\lstdefinelanguage{diff} {
    basicstyle=\ttfamily\small,
    morecomment=[f][\color{diffstart}]{@@},
    morecomment=[f][\color{diffincl}]{+\ },
    morecomment=[f][\color{diffrem}]{-\ },
  }

\begin{document}

% GitHub styles
\definecolor{keyword}{HTML}{CF222E}
\definecolor{comment}{HTML}{6E7781}
\definecolor{string}{HTML}{0A3069}
\definecolor{diffstart}{named}{Grey}
\definecolor{diffincl}{named}{Green}
\definecolor{diffrem}{named}{OrangeRed}

\lstset{
    commentstyle=\color{comment},
    keywordstyle=\color{keyword},
    stringstyle=\color{string},
    basicstyle=\ttfamily\small,
    breakatwhitespace=false,
    breaklines=true,
    captionpos=b,
    keepspaces=true,
    showspaces=false,
    showstringspaces=false,
    showtabs=false,
}

\title{MiniDecaf Stage 5 Report}

\maketitle

\section{实验内容}

\subsection{词法语法分析}

增加对于数组下标和数组维度的解析,对应 \texttt{VarRef} 和 \texttt{DeclStmt} 的修改:

\begin{lstlisting}[]
DeclStmt :
    ...
    | Type IDENTIFIER ArrayDims SEMICOLON
        { $$ = new ast::VarDecl($2, $1, $3, POS(@1)); }
    | Type IDENTIFIER ArrayDims ASSIGN LBRACE ExprList RBRACE SEMICOLON
        { $$ = new ast::VarDecl($2, $1, $3, $6, POS(@1)); }
    ;

ArrayDims :
    LBRACK ICONST RBRACK
        { $$ = new ast::DimList(); $$->append($2); }
    | LBRACK RBRACK
        { $$ = new ast::DimList(); $$->append(-1); }
    | ArrayDims LBRACK ICONST RBRACK
        { $$ = $1; $$->append($3); }
    ;

VarRef :
    ...
    | IDENTIFIER ArrayIndex
        { $$ = new ast::VarRef($1, $2, POS(@1)); }
    ;

ArrayIndex :
    LBRACK Expr RBRACK
        { $$ = new ast::ExprList(); $$->append($2); }
    | ArrayIndex LBRACK Expr RBRACK
        { $$ = $1; $$->append($3); }

\end{lstlisting}

另外,修改参数列表 \texttt{CommaSepParamList} 的定义以支持数组传参:

\begin{lstlisting}[]
CommaSepParamList :
    ...
    | Type IDENTIFIER ArrayDims
        { $$ = new ast::VarList(); $$->append(new ast::VarDecl($2, $1, $3, nullptr, POS(@1), true)); }
    ...
    | CommaSepParamList COMMA Type IDENTIFIER ArrayDims
        { $1->append(new ast::VarDecl($4, $3, $5, nullptr, POS(@3), true)); $$ = $1; }
    ;
\end{lstlisting}

\subsection{符号表构建}

在第一个 Pass,对于 \texttt{VarDecl} 节点,当声明为数组时,使用 \texttt{ArrayType}:

\begin{lstlisting}[language=c++]
void SemPass1::visit(ast::VarDecl *vdecl) {
    ...
    if (vdecl->isArray())
        t = new ArrayType(t, vdecl->dims);
    ...
}
\end{lstlisting}

\subsection{类型检查}

对于 \texttt{FuncCallExpr},需要检查所传参数与形参类型是否一致:

\begin{lstlisting}[language=c++]
void SemPass2::visit(ast::FuncCallExpr *e) {
    ...
    funcArgTypeIter = func->getType()->getArgList()->begin();
    for (auto arg = e->args->begin(); arg != e->args->end();
         ++arg, ++funcArgTypeIter) {
        (*arg)->accept(this);
        expect(*arg, *funcArgTypeIter);
        ++numArgs;
    }
    ...
}
\end{lstlisting}

对于 \texttt{VarRef},需要检查下标运算是否合法:

\begin{lstlisting}[language=c++]
void SemPass2::visit(ast::VarRef *ref) {
    ...
    if (ref->isArrayRef() && !v->getType()->isArrayType()) {
        issue(ref->getLocation(), new NotVariableError(v));
        goto issue_error_type;
    }

    ref->ATTR(sym) = (Variable *)v;

    if (ref->isArrayRef()) {
        ArrayType *at = dynamic_cast<ArrayType *>(v->getType());

        if (size_t(at->getDimList()->length()) !=
            ref->indexList->length()) {
            issue(ref->getLocation(), new BadIndexError());
            goto issue_error_type;
        }

        for (auto index = ref->indexList->begin();
             index != ref->indexList->end(); ++index) {
            (*index)->accept(this);
            expect(*index, BaseType::Int);
        }
        ref->ATTR(type) = at->getElementType();
        ref->ATTR(lv_kind) = ast::Lvalue::ARRAY_ELE;
    } else {
        ref->ATTR(type) = v->getType();
        ref->ATTR(lv_kind) = ast::Lvalue::SIMPLE_VAR;
    }
}
\end{lstlisting}

对于 \texttt{VarDecl},检查数组长度是否合法,注意对于参数列表中的声明可以省略第一维向量的长度。另外,检查初始化列表中的值是否为常数(只支持这种情况)。

\begin{lstlisting}[language=c++]
void SemPass2::visit(ast::VarDecl *decl) {
    if (decl->isArray()) {
        for (auto dim = decl->dims->begin(); dim != decl->dims->end(); ++dim) {
            if (*dim <= 0) {
                if (!decl->isParam() || dim != decl->dims->begin())
                    issue(decl->getLocation(), new ZeroLengthedArrayError());
            }
        }
    }

    if (decl->init) {
        if (decl->isArray())
            issue(decl->getLocation(), new NotArrayError());
        decl->init->accept(this);
        if (decl->ATTR(sym)->isGlobalVar() &&
            decl->init->getKind() != ast::ASTNode::INT_CONST) {
            issue(decl->getLocation(), new NotConstInitError());
        }
        if (!decl->init->ATTR(type)->compatible(decl->ATTR(sym)->getType()))
            issue(decl->getLocation(),
                    new IncompatibleError(decl->ATTR(sym)->getType(),
                                        decl->init->ATTR(type)));
    } else if (decl->init_list) {
        if (!decl->isArray())
            issue(decl->getLocation(), new NotArrayError());

        for (auto init = decl->init_list->begin();
                init != decl->init_list->end(); ++init) {
            (*init)->accept(this);
            if (decl->ATTR(sym)->isGlobalVar() &&
                (*init)->getKind() != ast::ASTNode::INT_CONST) {
                issue(decl->getLocation(), new NotConstInitError());
            }
        }
    }
}
\end{lstlisting}

\subsection{翻译为中间代码}

增加一个 TAC 类型:

\begin{itemize}
    \item \texttt{ALLOC dest, size}:在栈上分配 \texttt{size} 大小的空间,首地址赋值到 \texttt{dest}
\end{itemize}

翻译 \texttt{AssignExpr},对向数组赋值的表达式进行处理:

\begin{lstlisting}[language=c++]
void Translation::visit(ast::AssignExpr *s) {
    ...
    if (ref->ATTR(sym)->isGlobalVar()) {
        mind_assert(ref->ATTR(addr) != NULL);
        tr->genStore(ref->ATTR(addr), 0, s->e->ATTR(val));
    } else {
        if (isArrayRef) {
            tr->genStore(ref->ATTR(addr), 0, s->e->ATTR(val));
        } else {
            tr->genAssign(ref->ATTR(sym)->getTemp(), s->e->ATTR(val));
        }
    }
    ...
}
\end{lstlisting}

对于 \texttt{VarRef},我们计算全局变量或数组元素的地址,并附到节点的 \texttt{ATTR(addr)} 临时寄存器上:

\begin{lstlisting}[language=c++]
if (ref->ATTR(sym)->isGlobalVar()) {
    Temp addr = tr->genLoadSym(ref->ATTR(sym)->getLabel());
    if (isArrayRef) {
        ref->ATTR(addr) = tr->genAdd(addr, offset);
    } else {
        ref->ATTR(addr) = addr;
    }
} else {
    if (isArrayRef) {
        Temp base = ref->ATTR(sym)->getTemp();
        ref->ATTR(addr) = tr->genAdd(base, offset);
    } else {
        ref->ATTR(addr) = nullptr;
    }
}
\end{lstlisting}

其中计算数组 \texttt{offset} 时可以对常量表达式作优化,在编译期计算出偏移量:

\begin{lstlisting}[language=c++]
bool isConst = true;
int constOffset = 0;
int multiplier = 1;
auto dimIter = arrayType->getDimList()->rbegin();
for (auto iter = ref->indexList->rbegin();
        iter != ref->indexList->rend(); iter++, dimIter++) {
    (*iter)->accept(this);
    if ((*iter)->getKind() != ast::ASTNode::INT_CONST || !isConst) {
        if (isConst) {
            offset = tr->genLoadImm4(constOffset * 4);
            isConst = false;
        }
        Temp added = tr->genMul((*iter)->ATTR(val),
                                tr->genLoadImm4(multiplier * 4));
        offset = tr->genAdd(offset, added);
    } else {
        int val = dynamic_cast<ast::IntConst *>(*iter)->value;
        constOffset += val * multiplier;
    }
    multiplier *= *dimIter;
}

if (isConst) {
    offset = tr->genLoadImm4(constOffset * 4);
}
\end{lstlisting}
    
在翻译 \texttt{LvalueExpr} 时,利用在上面计算得到的 \texttt{VarRef} 的 \texttt{ATTR(addr)},可以很方便地计算 \texttt{ATTR(val)}。这里需要注意区分下标运算(取值)和数组传参(取地址),以及对全局变量和局部变量的区别处理:

\begin{lstlisting}[language=c++]
bool isArrayRef = ref->isArrayRef();
bool isArrayType = ref->ATTR(sym)->getType()->isArrayType();

if (ref->ATTR(sym)->isGlobalVar()) {
    if (isArrayRef || !isArrayType) {
        e->ATTR(val) = tr->genLoad(ref->ATTR(addr), 0);
    } else {
        e->ATTR(val) = tr->genLoadSym(ref->ATTR(sym)->getLabel());
    }
} else {
    if (isArrayRef) {
        e->ATTR(val) = tr->genLoad(ref->ATTR(addr), 0);
    } else {
        e->ATTR(val) = ref->ATTR(sym)->getTemp();
    }
}
\end{lstlisting}

翻译 \texttt{VarDecl} 时,增加对初值的处理:

\begin{lstlisting}[language=c++]
if (decl->ATTR(sym)->isGlobalVar()) {
    if (decl->init_list != nullptr) {
        defaultValues = new int[arrLength]();
        int i = 0;
        for (auto iter = decl->init_list->begin();
                iter != decl->init_list->end(); iter++, i++) {
            ast::IntConst *intConst =
                dynamic_cast<ast::IntConst *>(*iter);
            defaultValues[i] = intConst->value;
        }
    }
    tr->genDeclGlobVar(decl->ATTR(sym)->getLabel(), arrLength,
                       defaultValues);
} else {
    if (at != nullptr) {
        Temp temp = tr->genAlloc(arrLength);
        decl->ATTR(sym)->attachTemp(temp);

        if (decl->init_list != nullptr) {
            int i = 0;
            for (auto iter = decl->init_list->begin();
                 iter != decl->init_list->end(); iter++, i++) {
                (*iter)->accept(this);
                tr->genStore(temp, i * 4, (*iter)->ATTR(val));
            }

            int remaining = arrLength - decl->init_list->length();
            if (remaining > 0) {
                // Fill the rest with 0
                Temp len = tr->genLoadImm4(remaining);
                Temp startAddr = tr->genAdd(
                    temp, tr->genLoadImm4(decl->init_list->length() * 4));
                tr->genParam(startAddr, 0);
                tr->genParam(len, 1);
                tr->genParam(tr->genLoadImm4(0), 2);
                Label l = tr->getNewLabel();
                l->target = true;
                l->str_form = "fill_n";
                tr->genCall(l);
            }
        }
    } else {
        decl->ATTR(sym)->attachTemp(tr->getNewTempI4());
        if (decl->init != NULL) {
            decl->init->accept(this);
            tr->genAssign(decl->ATTR(sym)->getTemp(),
                          decl->init->ATTR(val));
        }
    }
}
\end{lstlisting}

\subsection{翻译为汇编代码}

对于 \texttt{ALLOC},首先修改 \texttt{SP} 在栈上分配空间,然后将修改后的 \texttt{SP} 赋值到目的寄存器即可。

\begin{lstlisting}[language=c++]
void RiscvDesc::emitAllocTac(tac::Tac *t) {
    if (!t->LiveOut->contains(t->op0.var))
        return;
    // Allocate memory on stack.
    // Modify SP to get enough space for the array.
    addInstr(RiscvInstr::ADDI, _reg[RiscvReg::SP],
             _reg[RiscvReg::SP], NULL, t->op1.ival * -4,
             EMPTY_STR, "allocate memory for array on stack");

    // Get current SP.
    int regIndex = getRegForWrite(t->op0.var, 0, 0,
                                  t->LiveOut);
    addInstr(RiscvInstr::MOVE, _reg[regIndex],
             _reg[RiscvReg::SP], NULL, 0, EMPTY_STR,
             "get current SP");
}
\end{lstlisting}

\section{思考题}

\begin{enumerate}
    \item \textbf{Step 11}:与现有实现类似(并未如文档所说统一分配内存),当遇到 \texttt{ALLOC} 指令时修改 \texttt{SP} 以分配空间并记录首地址,只不过空间大小是动态计算的而已。
    \item \textbf{Step 12}:计算地址(偏移量)不需要用到第一维的长度。
\end{enumerate}
\end{document}