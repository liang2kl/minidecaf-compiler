\documentclass[a4paper]{article}
\usepackage{listings}
\usepackage{ctex}
\usepackage[svgnames]{xcolor}
\usepackage{graphicx}
\usepackage{float}
\usepackage{parskip}

\setlength{\parskip}{6pt}

\author{梁业升 2019010547(计03)}

\lstdefinelanguage{diff} {
    basicstyle=\ttfamily\small,
    morecomment=[f][\color{diffstart}]{@@},
    morecomment=[f][\color{diffincl}]{+\ },
    morecomment=[f][\color{diffrem}]{-\ },
  }

\begin{document}

% GitHub styles
\definecolor{keyword}{HTML}{CF222E}
\definecolor{comment}{HTML}{6E7781}
\definecolor{string}{HTML}{0A3069}
\definecolor{diffstart}{named}{Grey}
\definecolor{diffincl}{named}{Green}
\definecolor{diffrem}{named}{OrangeRed}

\lstset{
    commentstyle=\color{comment},
    keywordstyle=\color{keyword},
    stringstyle=\color{string},
    basicstyle=\ttfamily\small,
    breakatwhitespace=false,
    breaklines=true,
    captionpos=b,
    keepspaces=true,
    showspaces=false,
    showstringspaces=false,
    showtabs=false,
}



\title{MiniDecaf Stage 2 Report}

\maketitle

\section{实验内容}

\subsection{词法语法分析}

局部变量声明:

\begin{lstlisting}[language=diff]
+ DeclStmt    : Type IDENTIFIER SEMICOLON
+                 { $$ = new ast::VarDecl($2, $1, POS(@1)); }
+             | Type IDENTIFIER ASSIGN Expr SEMICOLON
+                 { $$ = new ast::VarDecl($2, $1, $4, POS(@1)); }
+             ;
+ VarRef      : IDENTIFIER
+                 { $$ = new ast::VarRef($1, POS(@1)); }
+             ;
\end{lstlisting}    

变量引用:

\begin{lstlisting}[language=diff]
+ VarRef      : IDENTIFIER
+                 { $$ = new ast::VarRef($1, POS(@1)); }
+             ;
\end{lstlisting}    

赋值表达式

\begin{lstlisting}[language=diff]
  Expr        : ICONST
                  { $$ = new ast::IntConst($1, POS(@1)); }
+             | VarRef
+                 { $$ = new ast::LvalueExpr($1, POS(@1)); }
+             | VarRef ASSIGN Expr
+                 { $$ = new ast::AssignExpr($1, $3, POS(@2)); }
              | LPAREN Expr RPAREN
...
\end{lstlisting}

需要注意的是变量声明语句(\texttt{DeclStmt})不属于语句,所以不能放在 \texttt{Stmt} 中;但其可以和其他语句组成复合语句,因此
需要在 \texttt{StmtList} 中加上:

\begin{lstlisting}[language=diff]
  StmtList    : /* empty */
                  { $$ = new ast::StmtList(); }
              | StmtList Stmt
                  { $1->append($2); $$ = $1; }
+             | StmtList DeclStmt
+                 { $1->append($2); $$ = $1; }
\end{lstlisting}

另外,\texttt{IfStmt} 和 \texttt{IfExpr} 在框架中已给出,在此不再赘述。

\subsection{符号表构建}

在第一个 Pass,对于 \texttt{VarDecl} 节点,在当前作用域中添加符号表项,并给节点添加对应的符号:

\begin{lstlisting}[language=c++]
void SemPass1::visit(ast::VarDecl *vdecl) {
    Type *t = NULL;
    vdecl->type->accept(this);
    t = vdecl->type->ATTR(type);

    Variable *var = new Variable(vdecl->name, t, vdecl->getLocation());
    if (scopes->lookup(vdecl->name, vdecl->getLocation(), false) != nullptr) {
        issue(vdecl->getLocation(), new DeclConflictError(vdecl->name, var));
    } else {
        scopes->declare(var);
        vdecl->ATTR(sym) = var;
    }
}
\end{lstlisting}

\subsection{类型检查}

针对新增的表达式 \texttt{IfExpr} 增加类型检查(其余新增的表达式在框架中已给出):

\begin{lstlisting}[language=c++]
void SemPass2::visit(ast::IfExpr *e) {
    e->condition->accept(this);
    expect(e->condition, BaseType::Int);

    e->true_brch->accept(this);
    expect(e->true_brch, BaseType::Int);

    e->false_brch->accept(this);
    expect(e->false_brch, BaseType::Int);

    e->ATTR(type) = BaseType::Int;
}
\end{lstlisting}

\subsection{翻译为中间代码}

对于 \texttt{IfExpr},我们需要支持短路求值,因此需要用到条件跳转。例如,对于 \texttt{a = cond ? 1 : 0},对应的
三地址码如下(设 \texttt{a} 和 \texttt{cond} 的寄存器分别为 \texttt{T1} 和 \texttt{T2}):

\begin{lstlisting}
    JZERO  T2, L1
    ASSIGN T1, 1
    JUMP   L2
L1:
    ASSIGN T1, 0
L2:
\end{lstlisting}

对应的翻译代码如下:

\begin{lstlisting}[language=c++]
void Translation::visit(ast::IfExpr *e) {
    Label falseLabel = tr->getNewLabel();
    Label trueLabel = tr->getNewLabel();

    e->condition->accept(this);

    Temp temp = tr->getNewTempI4();

    tr->genJumpOnZero(falseLabel, e->condition->ATTR(val));
    e->true_brch->accept(this);
    tr->genAssign(temp, e->true_brch->ATTR(val));
    tr->genJump(trueLabel);

    tr->genMarkLabel(falseLabel);
    e->false_brch->accept(this);
    tr->genAssign(temp, e->false_brch->ATTR(val));

    tr->genMarkLabel(trueLabel);

    e->ATTR(val) = temp;
}
\end{lstlisting}

对于 \texttt{VarDecl},我们需要为变量(符号)分配一个临时寄存器(本阶段只考虑局部变量);如有初始化,则进行赋值:

\begin{lstlisting}[language=c++]
void Translation::visit(ast::VarDecl *decl) {
    if (decl->ATTR(sym)->isGlobalVar()) {
        mind_assert(false);
    } else {
        decl->ATTR(sym)->attachTemp(tr->getNewTempI4());
        if (decl->init != NULL) {
            decl->init->accept(this);
            tr->genAssign(decl->ATTR(sym)->getTemp(), decl->init->ATTR(val));
        }
    }
}
\end{lstlisting}

对于 \texttt{LvalueExpr},我们将对应符号的寄存器附在左值节点上:

\begin{lstlisting}[language=c++]
void Translation::visit(ast::LvalueExpr *e) {
    ast::VarRef *ref;
    switch (e->lvalue->getKind()) {
    case ast::ASTNode::VAR_REF:
        ref = dynamic_cast<ast::VarRef *>(e->lvalue);
        mind_assert(ref != NULL);
        ref->accept(this);

        if (ref->ATTR(sym)->isGlobalVar()) {
            mind_assert(false);
        } else {
            e->ATTR(val) = ref->ATTR(sym)->getTemp();
        }
        break;
    default:
        mind_assert(false);
    }
}
\end{lstlisting}

\subsection{生成机器代码}

本阶段需要新增的三地址码翻译只有 \texttt{ASSIGN},我们使用 \texttt{add rd, x0, rs1} 即可完成赋值:

\begin{lstlisting}[language=c++]
void RiscvDesc::emitAssignTac(Tac *t) {
    // eliminates useless assignments
    if (!t->LiveOut->contains(t->op0.var))
        return;

    int r0 = getRegForWrite(t->op0.var, 0, 0, t->LiveOut);
    int r1 = getRegForRead(t->op1.var, r0, t->LiveOut);
    addInstr(RiscvInstr::ADD, _reg[r0], _reg[RiscvReg::ZERO], _reg[r1], 0, EMPTY_STR, NULL);
}    
\end{lstlisting}

\section{思考题}

\begin{enumerate}
    \item \textbf{Step 5}:\begin{enumerate}
        \item \textbf{1}:\texttt{addi sp, sp, -26}
        \item \textbf{2}:定义:省略判断命名重复的步骤,直接覆盖即可;查找:无影响。
    \end{enumerate}
    \item \textbf{Step 6}:\begin{enumerate}
        \item \textbf{1}:Bison 默认使用 Shift。
        \item \textbf{2}:将两个 branch 分别执行完,再根据条件进行赋值。
    \end{enumerate}
\end{enumerate}


\end{document}