\documentclass[a4paper]{article}
\usepackage{listings}
\usepackage{ctex}
\usepackage[svgnames]{xcolor}
\usepackage{graphicx}
\usepackage{float}
\usepackage{parskip}

\setlength{\parskip}{6pt}

\author{梁业升 2019010547(计03)}

\lstdefinelanguage{diff} {
    basicstyle=\ttfamily\small,
    morecomment=[f][\color{diffstart}]{@@},
    morecomment=[f][\color{diffincl}]{+\ },
    morecomment=[f][\color{diffrem}]{-\ },
  }

\begin{document}

% GitHub styles
\definecolor{keyword}{HTML}{CF222E}
\definecolor{comment}{HTML}{6E7781}
\definecolor{string}{HTML}{0A3069}
\definecolor{diffstart}{named}{Grey}
\definecolor{diffincl}{named}{Green}
\definecolor{diffrem}{named}{OrangeRed}

\lstset{
  commentstyle=\color{comment},
  keywordstyle=\color{keyword},
  stringstyle=\color{string},
  basicstyle=\ttfamily\small,
  breakatwhitespace=false,
  breaklines=true,
  captionpos=b,
  keepspaces=true,
  showspaces=false,
  showstringspaces=false,
  showtabs=false,
}



\title{MiniDecaf Parser Stage Report}

\maketitle

\section{实验内容}

本次实验比较简单,根据文法实现相应的递归下降算法即可。

大部分的补全部分是基于 LL(1) 文法的递归下降,直接按照文法递归地调用对应的解析函数即可,如:

\begin{lstlisting}[language=c++]
if (next_token.type == TokenType::RETURN) {
    return p_Return();
} else if (next_token.type == TokenType::IF) {
    return p_If();
} else if (next_token.type == TokenType::LPAREN) {
    lookahead(TokenType::LPAREN);
    ast::Expr *expr = p_Expression();
    lookahead(TokenType::RPAREN);
    lookahead(TokenType::SEMICOLON);
    return new ast::ExprStmt(expr, expr->getLocation());
} else if (next_token.type == TokenType::LBRACE) {
    return p_Block();
} else if (next_token.type == TokenType::SEMICOLON) {
    lookahead(TokenType::SEMICOLON);
    return new ast::EmptyStmt(next_token.loc);
}
\end{lstlisting}

而对于包含左递归的文法,根据提示,转换为 EBNF 文法,并使用 \texttt{while} 循环进行解析:

\begin{lstlisting}[language=c++]
while (next_token.type == TokenType::OR) {
    Token Or = lookahead();
    ast::Expr *operand2 = p_LogicalAnd();
    node = new ast::OrExpr(node, operand2, Or.loc);
}
\end{lstlisting}

所有的待填空部分均为这两种情况之一,在代码中均有体现,在此不再赘述。

\section{思考题}

1. 原文法:

\begin{lstlisting}
additive : additive '+' multiplicative
         | additive '-' multiplicative
         | multiplicative
\end{lstlisting}

不含左递归的文法:

\begin{lstlisting}
additive : multiplicative A
A        : '+' multiplicative A
         | '-' multiplicative A
         | epsilon
\end{lstlisting}

2. 在递归解析某一条文法规则时,如果下一个 token 匹配错误,则继续消耗 token,直到遇到文法左侧
非终结符的同步集合中的元素,则视为开始下一个文法规则的匹配。

例如:

\begin{lstlisting}[language=c++]
int a = ;
int b = 0;
\end{lstlisting}

第一行对应的文法为:

\begin{lstlisting}
DeclStmt : Type IDENTIFIER ASSIGN Expr SEMICOLON
\end{lstlisting}

\texttt{DeclStmt} 同步集合中包含关键字 \texttt{int},因此遇到第二行的 \texttt{int} 时即视为错误已解决。

\end{document}